

\documentclass{standalone}
\usepackage{tikz}

\usetikzlibrary{math}
\usetikzlibrary{calc}

\begin{document}
\definecolor{limegreen}{rgb}{0.2, 0.8, 0.2}
\newcommand{\tikzAngleOfLine}{\tikz@AngleOfLine}
    \def\tikz@AngleOfLine(#1)(#2)#3{%
    \pgfmathanglebetweenpoints{%
        \pgfpointanchor{#1}{center}}{%
        \pgfpointanchor{#2}{center}}
    \pgfmathsetmacro{#3}{\pgfmathresult}%
}
\tikzmath{
    \ballax = 4.134;
    \ballay = 3.000; 
    \ballbx = 5.866;
    \ballby = 2.000;
    \vecx = \ballbx - cos(60) * 1.5;
    \vecy = \ballby + sin(60) * 1.5;
    \dashx = \ballax + cos(60) * 1.75;
    \dashy = \ballay - sin(60) * 1.75;
}
\begin{tikzpicture}
    \fill[limegreen] (0,0) rectangle (10,5);
    \fill[red] (\ballax, \ballay) circle (1);
    \fill[white] (\ballbx, \ballby) circle (1);
    \draw [->, line width=2pt] (\ballbx, \ballby) -- (\vecx, \vecy);
    \draw [dotted, line width=2pt] (\ballax, \ballay) -- (\dashx, \dashy);
    \draw [dotted, line width=2pt] (\ballbx, \ballby) -- node[below] {$e$} (\dashx, \dashy);
\end{tikzpicture}

\end{document}
