
\documentclass[12pt]{article}
\usepackage[a4paper, total={6.5in, 9in}]{geometry}
\usepackage{import}

\import{./}{macros}


\title{
    % UW Mech/Tron Eng Logo
    \includegraphics[width=\linewidth]{resources/uwaterloo_mechanical_and_mechatronics_engineering/UWaterloo_Mechanical_and_Mechatronics_Engineering/PNG/UWaterloo_Mechanical_Mechatronics_Eng_Logo_horiz_rgb.png}
    \\[1cm]
    \underline{\bf{ME 212 Billards Project Report}}
}
\author{
    Alex Roman\\
    Austin W. Milne
}
\date{July 8, 2021}


\begin{document}
\maketitle
\vfill
\begin{abstract}
    Project report for ME 212 in Spring 2021. Using MatLab to simulation collisions of pool balls 
    with friction and restitution.
\end{abstract}
\newpage

\tableofcontents
\listoftables
\listoffigures
\lstlistoflistings

\newpage

% \begin{abstract}
% Report 
% \end{abstract}

\section{Part 1 - Collision of 2 Balls}
\subsection{Given Information}
The problem is given to solve for the final velocities of 2 balls with arbitrary mass and initial velocity
after colliding at a certain offset distance. Figure \ref{P1_diag} shows the setup of the 2 balls for the collision.

\begin{figure}[H]
    \centering
    \includestandalone[width=10cm]{resources/tikz/part_1_diagram}
    \captionof{figure}{Part 1 Setup}\label{P1_diag}
\end{figure}

\subsection{Physics of The Collision}

Since it can be assume that there are no external forces during the collision, The problem becomes a 
straightforward conservation of momentum problem. The velocites can be split for each ball into the direction
normal and tangent to the collision. Each part can be solved seperately. First, create a system of equations
for the normal velocites before and after the impact. Equations \ref{eqn:pt1_momentum} and \ref{eqn:pt1_restitution} show the 
equations for conservation of momentum and for impact restitution, respectively.

\begin{equation}
    \label{eqn:pt1_momentum}
    v_a m_a + v_b m_b = v_a' m_a + v_b' m_b
\end{equation}

\begin{equation}
    \label{eqn:pt1_restitution}
    e_{\text{rest.}} = \frac{v_a' - v_b'}{v_b - v_a}
\end{equation}

Solving this system of equations gives the normal velocities of the 2 balls after the collision, $v_a'$ and $v_b'$. For tangent
velocities, due to friction, energy loses need to be accounted for in the collision. Since the change in momentum
in the normal direction is the intergal of the force over the impact time, the change in tangent velocity can be
derived from the change in normal momentum. Equation \ref{eqn:tangent_fric} shows the derivation for a balls tangent
velocity after impact.

\begin{equation}
    \label{eqn:tangent_fric}
    \begin{aligned}
        mv + \int_{t_1}^{t_2} Fdt &= mv' \\
        mv_t + \int_{t_1}^{t_2} F_f dt &= mv_t' \\
        mv_t + \mu\left(mv_n' - mv_n\right) &= mv_t' \ \Leftarrow \text{Using $F_f = \mu_k F_n$}\\
        v_t + \mu\left(v_n' - v_n\right) &= v_t'
    \end{aligned}
\end{equation}

\subsection{Application of Physics}
In order to apply the equations above, the normal and tangent velocities of the balls need to be determined. Basic
trigonometery can be used to determine the relative velocities. In this case, it is difficult to determine the 
normal and tangent values from the velocity vector and the $e$ value alone. To make it easier, Figure 
\ref{P1_diag_exp} shows construction lines added for easier computation.

\begin{figure}[H]
    \centering
    \includestandalone[width=10cm]{resources/tikz/part_1_diagram_expanded}
    \captionof{figure}{Part 1 Construction Lines}\label{P1_diag_exp}
\end{figure}

Using the construction lines above, the below equations can be derived for the tangent and normal velocities of the
moving ball. Equation \ref{eqn:part_1_thetas} shows the derivation of the impact angle and relative angles for vector rotation.

\begin{equation}
    \label{eqn:part_1_thetas}
    \begin{gathered}
        h = 2r \\
        \theta = \sin^{-1} \left(\frac{e}{h}\right) = \sin^{-1} \left(\frac{e}{2r}\right)\\
        \theta_\text{velocity} = \tan^{-1}\left(\frac{v_x}{v_y}\right)\\
        \theta_\text{impact} = \theta_\text{velocity} - \theta =  \tan^{-1}\left(\frac{v_x}{v_y}\right) - \sin^{-1} \left(\frac{e}{2r}\right)\\
    \end{gathered}
\end{equation}

Equation \ref{eqn:part_1_impact_v} shows the derivation for the normal and tangent velocities relative to the collision by
way of matrix-based vector rotation.

\begin{equation}
    \label{eqn:part_1_impact_v}
    \begin{gathered}
        \phi = \theta_\text{impact} - \theta_\text{velocity}\\
        R = 
        \left[
        \begin{matrix}
            \cos(\phi) & -\sin{\phi}\\
            \sin(\phi) & \cos(\phi)\\
        \end{matrix}
        \right]\\
        v_{a_\text{norm,tang}} = R \times v_a
    \end{gathered}
\end{equation}

Having the normal and tangent velocities for the collision, the below set Equation set \ref{eqn:part_1_together} can be applied to get the
final normal and tangent velocities for the both ball A (red) and ball b (white).

\begin{equation}
    \label{eqn:part_1_together}
    \begin{gathered}
        v_{a_n} m_a + v_{b_n} m_b = v_{a_n}' m_a + v_{b_n}' m_b\\
        e_\text{rest.} = \frac{v_{a_n}' - v_{b_n}'}{v_{b_n} - v_{a_b}}\\
        v_{a_t}' = v_{a_t} + \mu \left(v_{a_n}' - v_{a_n}\right)\\
        v_{b_t}' = v_{b_t} + \mu \left(v_{b_n}' - v_{b_n}\right)
    \end{gathered}
\end{equation}

Finally, to get the final velocities of the 2 balls in the $x$ $y$ reference plane, another vector rotation
needs to be done to revert to the $xy$ coordinate system. Equation \ref{eqn:part_1_rotate_back} shows this vector rotation.

\begin{equation}
    \label{eqn:part_1_rotate_back}
    \begin{gathered}
        \phi = \theta_\text{velocity} - \theta_\text{impact}\\
        R = 
        \left[
        \begin{matrix}
            \cos(\phi) & -\sin{\phi}\\
            \sin(\phi) & \cos(\phi)\\
        \end{matrix}
        \right]\\
        v_a = R \times v_{a_\text{norm,tang}} \\
        v_b = R \times v_{b_\text{norm,tang}}
    \end{gathered}
\end{equation}

\pagebreak
\section{Appendix}

\centering{
    \captionof{table}{Starting Positions}\label{start_pos}
    \begin{tabular}{l|r|r}
        \bfseries Ball \# & \bfseries X pos & \bfseries Y pos
        \csvreader[head to column names]{project_information/Initial_Positions.csv}{}
        {\\\hline \Ball & \x & \y}
    \end{tabular}
}

\begin{table}[H]
    \centering
    \captionof{table}{Simulation Snapshots}\label{sim_snaps}
    \begin{tabu}to \textwidth {X[c]X[c]}
        \includegraphics[width=75mm]{out/matlab/images/Simulation_1_start.png}\captionof{figure}{Simulation 1 Start}
        &\includegraphics[width=75mm]{out/matlab/images/Simulation_1_end.png}\captionof{figure}{Simulation 1 End} \\
        \includegraphics[width=75mm]{out/matlab/images/Simulation_2_start.png}\captionof{figure}{Simulation 2 Start}
        &\includegraphics[width=75mm]{out/matlab/images/Simulation_2_end.png}\captionof{figure}{Simulation 2 End} \\
        \includegraphics[width=75mm]{out/matlab/images/Simulation_3_start.png}\captionof{figure}{Simulation 3 Start}
        &\includegraphics[width=75mm]{out/matlab/images/Simulation_3_end.png}\captionof{figure}{Simulation 3 End} \\
        \includegraphics[width=75mm]{out/matlab/images/Simulation_4_start.png}\captionof{figure}{Simulation 4 Start}
        &\includegraphics[width=75mm]{out/matlab/images/Simulation_4_end.png}\captionof{figure}{Simulation 4 End} \\
    \end{tabu}
\end{table}
\pagebreak
\lstinputlisting[ title=\texttt{\scriptsize output.log}, caption=Script Output | output.log ]{./out/matlab/output.log}
\pagebreak
\lstinputlisting[ title=\texttt{\scriptsize BilliardsCode.m}, caption=MatLab Script | BilliardsCode.m ]{./BilliardsCode.m}
\pagebreak
\lstinputlisting[ title=\texttt{\scriptsize PoolBall.m}, caption=PoolBall Class Definition | PoolBall.m ]{./PoolBall.m}


\bibliographystyle{abbrv}
\bibliography{main}

\end{document}
